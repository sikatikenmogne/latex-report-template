% =====================================================
% TUTORIEL COMPLET - TEMPLATE RAPPORT DE STAGE
% =====================================================

\documentclass{internshipreport}

% =====================================================
% CONFIGURATION MÉTADONNÉES (ÉTAPE 1)
% =====================================================

% Configuration du rapport (À personnaliser)
\reporttitle{Tutorial: Professional Report Template Features}
\reportsubtitle{Complete Guide to LaTeX Template Customization}
\reporttype{TUTORIAL REPORT}

% Informations étudiant
\reportauthor{John Doe Student}
\studentid{ST2024001}
\program{Computer Science Engineering}
\academicyear{2024-2025}
\internshipperiod{January 20 - June 20, 2025}

% Institution académique
\university{University of Technology}
\department{Computer Science Department}
\academicsupervisor{Dr. Academic Supervisor}
\supervisorposition{Professor in Software Engineering}

% Entreprise d'accueil
\company{Tech Innovation Corp}
\companydepartment{Software Development Department}
\companytutor{Jane Smith}
\tutorposition{Senior Software Architect}

% Informations supplémentaires
\defensedate{July 15, 2025}

% =====================================================
% CHARGEMENT DES MODULES DE CONFIGURATION
% =====================================================

% Chargement des configurations modulaires
% =====================================================
% PACKAGES ET CONFIGURATION DE BASE
% =====================================================

% Encodage et langue
\usepackage[utf8]{inputenc}
\usepackage[english]{babel}
\usepackage[T1]{fontenc}
\usepackage{lmodern}

% Géométrie et mise en page
\usepackage[left=2.54cm,right=2.54cm,top=2.54cm,bottom=2.54cm]{geometry}
\usepackage{fancyhdr}
\usepackage{titlesec}
\usepackage{setspace}
\usepackage{indentfirst}
\usepackage{tocloft}

% Graphiques et images
\usepackage{graphicx}
\usepackage{wrapfig}
\usepackage{subfig}
\usepackage{caption}
\usepackage{subcaption}
\usepackage[table]{xcolor}
\usepackage{colortbl}
\usepackage{float}
\usepackage{tcolorbox}
\tcbuselibrary{most}

% Tableaux avancés
\usepackage{array}
\usepackage{tabularx}
\usepackage{longtable}
\usepackage{multirow}
\usepackage{multicol}
\usepackage{booktabs}

% Listes et énumérations
\usepackage{enumitem}

% Code et verbatim
\usepackage{listings}
\usepackage{fancyvrb}

% Mathématiques
\usepackage{amsmath}
\usepackage{amsfonts}
\usepackage{amssymb}

% Bibliographie et références
\usepackage[backend=biber,style=apa,sorting=none]{biblatex}
\addbibresource{content/backmatter/bibliography.bib}
\usepackage[hidelinks]{hyperref}
\usepackage{url}
\usepackage{cleveref}

% Police professionnelle
\usepackage{times}
\usepackage{mathptmx}

% Divers
\usepackage{lipsum}
\usepackage{afterpage}

% Configuration de fancyhdr
\setlength{\headheight}{14.5pt}
    % Packages additionnels
% =====================================================
% SCHÉMA DE COULEURS PERSONNALISÉ POUR LE TUTORIEL
% =====================================================

% ===== COULEURS PRINCIPALES DU PROJET =====

% Couleurs de marque (personnalisables selon l'entreprise)
\definecolor{brandprimary}{RGB}{25,118,210}        % Bleu corporate moderne
\definecolor{brandsecondary}{RGB}{46,125,50}       % Vert professionnel
\definecolor{brandaccent}{RGB}{255,112,67}         % Orange accent dynamique

% Variantes des couleurs principales
\definecolor{brandprimarylight}{RGB}{144,202,249}  % Bleu clair
\definecolor{brandsecondarylight}{RGB}{129,199,132} % Vert clair
\definecolor{brandaccentlight}{RGB}{255,183,77}    % Orange clair

\definecolor{brandprimarydark}{RGB}{13,71,161}     % Bleu foncé
\definecolor{brandsecondarydark}{RGB}{27,94,32}    % Vert foncé
\definecolor{brandaccentdark}{RGB}{230,74,25}      % Orange foncé

% ===== COULEURS NEUTRES PROFESSIONNELLES =====

% Échelle de gris moderne
\definecolor{neutraldark}{RGB}{33,33,33}           % Noir corporate
\definecolor{neutralmedium}{RGB}{97,97,97}         % Gris moyen
\definecolor{neutrallight}{RGB}{158,158,158}       % Gris clair
\definecolor{neutrallighter}{RGB}{224,224,224}     % Gris très clair
\definecolor{neutralwhite}{RGB}{255,255,255}       % Blanc pur

% Couleurs de fond
\definecolor{backgroundprimary}{RGB}{250,250,250}  % Fond principal
\definecolor{backgroundsecondary}{RGB}{245,245,245} % Fond secondaire
\definecolor{backgroundcard}{RGB}{255,255,255}     % Fond cartes/boîtes

% ===== COULEURS SÉMANTIQUES =====

% États et notifications
\definecolor{successcolor}{RGB}{76,175,80}         % Succès (vert)
\definecolor{warningcolor}{RGB}{255,152,0}         % Avertissement (orange)
\definecolor{errorcolor}{RGB}{244,67,54}           % Erreur (rouge)
\definecolor{infocolor}{RGB}{33,150,243}           % Information (bleu)

% Variantes claires pour les fonds
\definecolor{successlight}{RGB}{232,245,233}       % Fond succès
\definecolor{warninglight}{RGB}{255,243,224}       % Fond avertissement
\definecolor{errorlight}{RGB}{255,235,238}         % Fond erreur
\definecolor{infolight}{RGB}{227,242,253}          % Fond information

% ===== COULEURS POUR LE CODE SOURCE =====

% Syntaxe highlighting moderne
\definecolor{codebackground}{RGB}{248,249,250}     % Fond code
\definecolor{codeborder}{RGB}{218,220,224}         % Bordure code
\definecolor{codekeyword}{RGB}{215,58,73}          % Mots-clés
\definecolor{codecomment}{RGB}{106,153,85}         % Commentaires
\definecolor{codestring}{RGB}{3,102,214}           % Chaînes
\definecolor{codenumber}{RGB}{0,92,197}            % Nombres
\definecolor{codefunction}{RGB}{121,93,163}        % Fonctions
\definecolor{codevariable}{RGB}{227,98,9}          % Variables

% ===== COULEURS POUR LES TABLEAUX =====

% Style moderne pour tableaux
\definecolor{tableheadercolor}{RGB}{37,99,235}     % En-têtes (bleu)
\definecolor{tableheadertext}{RGB}{255,255,255}    % Texte en-têtes
\definecolor{tablealternatecolor}{RGB}{248,250,252} % Lignes alternées
\definecolor{tablehovercolor}{RGB}{219,234,254}    % Survol (bleu clair)
\definecolor{tableborder}{RGB}{203,213,225}        % Bordures

% ===== COULEURS POUR LES GRAPHIQUES =====

% Palette accessible pour graphiques (respecte les standards d'accessibilité)
\definecolor{chartblue}{RGB}{54,162,235}           % Bleu graphique
\definecolor{chartred}{RGB}{255,99,132}            % Rouge graphique
\definecolor{chartgreen}{RGB}{75,192,192}          % Vert graphique
\definecolor{chartyellow}{RGB}{255,205,86}         % Jaune graphique
\definecolor{chartpurple}{RGB}{153,102,255}        % Violet graphique
\definecolor{chartorange}{RGB}{255,159,64}         % Orange graphique

% ===== COULEURS POUR LA GESTION DE PROJET =====

% États des tâches
\definecolor{todostatus}{RGB}{156,163,175}         % À faire (gris)
\definecolor{inprogressstatus}{RGB}{59,130,246}    % En cours (bleu)
\definecolor{reviewstatus}{RGB}{245,158,11}        % En révision (orange)
\definecolor{donestatus}{RGB}{34,197,94}           % Terminé (vert)
\definecolor{blockedstatus}{RGB}{239,68,68}        % Bloqué (rouge)

% Priorités
\definecolor{prioritylow}{RGB}{156,163,175}        % Basse (gris)
\definecolor{prioritymedium}{RGB}{245,158,11}      % Moyenne (orange)
\definecolor{priorityhigh}{RGB}{239,68,68}         % Haute (rouge)
\definecolor{prioritycritical}{RGB}{147,51,234}    % Critique (violet)

% ===== CONFIGURATION DES LIENS HYPERTEXTE =====

\hypersetup{
    colorlinks=true,
    linkcolor=neutraldark,          % Liens internes
    citecolor=brandsecondary,        % Citations
    filecolor=brandaccent,           % Fichiers
    urlcolor=brandprimary,           % URLs
    anchorcolor=brandprimarydark,    % Ancres
    menucolor=brandprimary,          % Menus PDF
    runcolor=brandaccent,            % Exécutables
    bookmarksnumbered=true,
    bookmarksopen=true,
    bookmarksopenlevel=1,
    pdfstartview=FitH,
    pdfpagelayout=OneColumn,
    pdfdisplaydoctitle=true
}

% ===== COULEURS POUR L'IMPRESSION =====

% Alternatives haute-contraste pour l'impression
\definecolor{printblack}{RGB}{0,0,0}               % Noir impression
\definecolor{printgray}{RGB}{64,64,64}             % Gris impression
\definecolor{printlightgray}{RGB}{128,128,128}     % Gris clair impression

% Commande pour basculer en mode impression
\newif\ifprintmode
\printmodefalse  % Mode écran par défaut

% Redéfinition conditionnelle des couleurs pour l'impression
\ifprintmode
    \definecolor{brandprimary}{RGB}{0,0,0}
    \definecolor{brandsecondary}{RGB}{64,64,64}
    \definecolor{brandaccent}{RGB}{128,128,128}
\fi
      % Couleurs personnalisées
% =====================================================
% COMMANDES PERSONNALISÉES POUR LE TUTORIEL
% =====================================================

% ===== COMMANDES SPÉCIFIQUES AU PROJET =====

% Références d'entreprise et organisations
\newcommand{\companyref}[1]{\textcolor{primarycolor}{\textbf{#1}}}
\newcommand{\universityref}[1]{\textcolor{secondarycolor}{\textbf{#1}}}

% Nom du projet et technologies principales
\newcommand{\projectref}{\textbf{\color{brandprimary}\projectname}}
\newcommand{\mainframework}{\framework{\maintech}}
\newcommand{\backend}{\framework{\backendtech}}
\newcommand{\maindatabase}{\database{\databasetech}}
\newcommand{\programming}[1]{\texttt{\color{primarycolor}#1}}

% ===== COMMANDES POUR LES MÉTHODOLOGIES =====

% Méthodologies de développement
\newcommand{\agile}{\methodology{Agile}}
\newcommand{\scrum}{\methodology{Scrum}}
\newcommand{\kanban}{\methodology{Kanban}}
\newcommand{\devops}{\methodology{DevOps}}

% Éléments Scrum spécifiques
\newcommand{\sprintduration}[1]{\duration{#1-week sprint}}
\newcommand{\userstoryref}[1]{\textbf{\color{brandaccent}US-#1}}
\newcommand{\taskref}[1]{\textbf{\color{brandsecondary}TASK-#1}}

% ===== COMMANDES POUR LES TESTS =====

% Types de tests
\newcommand{\unittest}{\testcase{Unit Test}}
\newcommand{\integrationtest}{\testcase{Integration Test}}
% \newcommand{\e2etest}{\testcase{End-to-End Test}}
\newcommand{\performancetest}{\testcase{Performance Test}}

% Couverture de tests
\newcommand{\codecoverage}[1]{\metric{#1\% coverage}}
\newcommand{\testpass}[1]{\textcolor{brandsuccess}{\textbf{#1 passed}}}
\newcommand{\testfail}[1]{\textcolor{branddanger}{\textbf{#1 failed}}}

% ===== COMMANDES POUR LES PERFORMANCES =====

% Métriques de performance
\newcommand{\loadtime}[1]{\metric{#1ms load time}}
\newcommand{\responsetime}[1]{\metric{#1ms response}}
\newcommand{\throughput}[1]{\metric{#1 req/sec}}
\newcommand{\concurrent}[1]{\metric{#1 concurrent users}}

% Améliorations de performance
\newcommand{\speedup}[1]{\improvement{#1× faster}}
\newcommand{\reduction}[1]{\improvement{#1\% reduction}}
\newcommand{\increase}[1]{\improvement{#1\% increase}}

% ===== COMMANDES POUR LA SÉCURITÉ =====

% Éléments de sécurité
\newcommand{\authentication}{\important{Authentication}}
\newcommand{\authorization}{\important{Authorization}}
\newcommand{\encryption}{\important{Encryption}}
\newcommand{\validation}{\important{Input Validation}}

% Types d'utilisateurs et rôles
% \newcommand{\userrole}[1]{\role{#1}}
\newcommand{\adminrole}{\role{Administrator}}
\newcommand{\userrole}{\role{User}}
\newcommand{\guestrole}{\role{Guest}}

% ===== COMMANDES POUR LES APIs =====

% Méthodes HTTP
\newcommand{\httpget}{\httpmethod{GET}}
\newcommand{\httppost}{\httpmethod{POST}}
\newcommand{\httpput}{\httpmethod{PUT}}
\newcommand{\httpdelete}{\httpmethod{DELETE}}

% Codes de statut HTTP
\newcommand{\statusok}{\statuscode{200 OK}}
\newcommand{\statuscreated}{\statuscode{201 Created}}
\newcommand{\statusnotfound}{\statuscode{404 Not Found}}
\newcommand{\statuserror}{\statuscode{500 Internal Server Error}}

% Endpoints API
\newcommand{\apiendpoint}[1]{\endpoint{/api/v1/#1}}

% ===== COMMANDES POUR LA DOCUMENTATION =====

% Types de documentation
\newcommand{\apidoc}{\deliverable{API Documentation}}
\newcommand{\userdoc}{\deliverable{User Documentation}}
\newcommand{\techdoc}{\deliverable{Technical Documentation}}

% Outils de documentation
\newcommand{\swagger}{\software{Swagger}}
\newcommand{\postman}{\software{Postman}}
\newcommand{\jsdoc}{\software{JSDoc}}

% ===== COMMANDES POUR LES OUTILS =====

% Outils de développement
\newcommand{\vscode}{\software{Visual Studio Code}}
\newcommand{\git}{\software{Git}}
\newcommand{\github}{\software{GitHub}}
\newcommand{\docker}{\software{Docker}}
\newcommand{\kubernetes}{\software{Kubernetes}}

% Services cloud
\newcommand{\aws}{\software{Amazon Web Services}}
\newcommand{\azure}{\software{Microsoft Azure}}
\newcommand{\gcp}{\software{Google Cloud Platform}}

% ===== COMMANDES POUR LES DATES ET DURÉES =====

% Phases du projet
\newcommand{\phase}[1]{\milestone{Phase #1}}
\newcommand{\iteration}[1]{\milestone{Iteration #1}}
\newcommand{\release}[1]{\milestone{Release #1}}

% Durées typiques
\newcommand{\oneweek}{\duration{1 week}}
\newcommand{\twoweeks}{\duration{2 weeks}}
\newcommand{\onemonth}{\duration{1 month}}

% ===== COMMANDES POUR LES RÉFÉRENCES =====

% Références aux livrables
\newcommand{\deliverableref}[1]{Deliverable~\ref{deliv:#1}}
\newcommand{\milestoneref}[1]{Milestone~\ref{mile:#1}}
\newcommand{\requirementref}[1]{Requirement~\ref{req:#1}}

% Références aux tests
\newcommand{\testcaseref}[1]{Test Case~\ref{test:#1}}
\newcommand{\testsuiteref}[1]{Test Suite~\ref{suite:#1}}

% ===== COMMANDES CONDITIONNELLES =====

% Compilation conditionnelle
\newif\ifshowdetails
\showdetailstrue  % Afficher les détails par défaut

\newcommand{\detailsonly}[1]{\ifshowdetails#1\fi}
\newcommand{\summaryonly}[1]{\ifshowdetails\else#1\fi}

% Version du document
\newcommand{\version}[1]{\textit{\color{neutralmedium}v#1}}
\newcommand{\updated}[1]{\textit{\color{neutralmedium}Updated: #1}}

    % Commandes personnalisées
% =====================================================
% CONFIGURATION AVANCÉE DU STYLE
% =====================================================

% ===== TYPOGRAPHIE PERSONNALISÉE =====

% Interligne (choix: \singlespacing, \onehalfspacing, \doublespacing)
\onehalfspacing  % Interligne 1.5 (recommandé pour les rapports)

% Espacement des paragraphes
\setlength{\parskip}{8pt plus 2pt minus 1pt}    % 8pt entre paragraphes
\setlength{\parindent}{0pt}                     % Pas d'indentation première ligne

% Marges personnalisées (optionnel - déjà défini dans la classe)
% \geometry{
%     top=2.5cm,
%     bottom=2.5cm,
%     left=3cm,
%     right=2.5cm,
%     headheight=15pt
% }

% ===== PERSONNALISATION DES TITRES =====

% Espacement autour des titres
\titlespacing*{\chapter}{0pt}{-10pt}{25pt}      % Chapitre: espace avant/après
\titlespacing*{\section}{0pt}{18pt}{12pt}       % Section: espace avant/après
\titlespacing*{\subsection}{0pt}{14pt}{8pt}     % Sous-section: espace avant/après
\titlespacing*{\subsubsection}{0pt}{12pt}{6pt}  % Sous-sous-section

% Couleurs des titres (personnalisable)
\titleformat{\chapter}[display]
    {\normalfont\huge\bfseries\color{brandprimary}}
    {\chaptertitlename\ \thechapter}{20pt}{\Huge\MakeUppercase}

\titleformat{\section}
    {\normalfont\Large\bfseries\color{brandsecondary}}
    {\thesection}{1em}{}

\titleformat{\subsection}
    {\normalfont\large\bfseries\color{neutraldark}}
    {\thesubsection}{1em}{}

% ===== CONFIGURATION DES LISTES =====

% Listes à puces - niveau 1
\setlist[itemize,1]{
    label=\textcolor{neutraldark}{\textbullet},
    leftmargin=20pt,
    itemsep=3pt,
    parsep=1pt,
    topsep=6pt,
    partopsep=0pt
}

% Listes à puces - niveau 2
\setlist[itemize,2]{
    label=\textcolor{neutraldark}{\textendash},
    leftmargin=35pt,
    itemsep=2pt,
    parsep=0pt,
    topsep=3pt
}

% Listes numérotées
\setlist[enumerate,1]{
    leftmargin=20pt,
    itemsep=3pt,
    parsep=1pt,
    topsep=6pt
}

% ===== STYLE DES TABLEAUX =====

% Espacement global des tableaux
\renewcommand{\arraystretch}{1.4}  % Hauteur des lignes
\setlength{\tabcolsep}{10pt}       % Espacement des colonnes

% Style des en-têtes de tableaux
\renewcommand{\tableheadercell}[1]{%
    \cellcolor{tableheadercolor}\textbf{\color{neutraldark}#1}%
}

% ===== CONFIGURATION DES FIGURES =====

% Positionnement par défaut des figures
\renewcommand{\topfraction}{0.9}       % Max 90% de la page pour figures en haut
\renewcommand{\bottomfraction}{0.9}    % Max 90% de la page pour figures en bas
\renewcommand{\textfraction}{0.1}      % Min 10% de la page pour le texte
\renewcommand{\floatpagefraction}{0.8} % Min 80% pour page de figures

% Style des légendes
\captionsetup[figure]{
    font={small,times},                    % Police sans-serif petite
    labelfont={bf,color=neutraldark},  % Label en gras coloré
    textfont={color=neutraldark},       % Texte en couleur neutre
    margin=20pt,                        % Marge des légendes
    skip=12pt,                          % Espace avec la figure
    position=bottom                     % Position sous la figure
}

\captionsetup[table]{
    font={small,times},
    labelfont={bf,color=neutraldark},
    textfont={color=neutraldark},
    margin=20pt,
    skip=8pt,
    position=top                        % Position au-dessus du tableau
}

% ===== STYLE DU CODE SOURCE =====

% Configuration globale pour tous les listings
\lstset{
    backgroundcolor=\color{codebackground},
    basicstyle=\footnotesize\ttfamily\color{neutraldark},
    commentstyle=\color{codecomment}\itshape,
    keywordstyle=\color{codekeyword}\bfseries,
    stringstyle=\color{codestring},
    numberstyle=\tiny\color{neutralmedium},
    numbers=left,
    numbersep=12pt,
    frame=single,
    rulecolor=\color{codeborder},
    frameround=tttt,
    framesep=10pt,
    xleftmargin=20pt,
    xrightmargin=10pt,
    breaklines=true,
    breakatwhitespace=false,
    tabsize=2,
    showspaces=false,
    showstringspaces=false,
    showtabs=false,
    captionpos=b
}

% ===== EN-TÊTES ET PIEDS DE PAGE =====

% Style pour les pages normales
\fancypagestyle{mainmatter}{
    \fancyhf{}
    \fancyhead[C]{\footnotesize\textcolor{neutralmedium}{\MakeUppercase{\@reporttitle}}}
    \fancyfoot[R]{\footnotesize\textcolor{neutralmedium}{%
        \leftmark \quad \textbf{|} \quad \textbf{\thepage}%
    }}
    \renewcommand{\headrulewidth}{0pt}
    \renewcommand{\footrulewidth}{0.5pt}
    \renewcommand{\footrule}{\color{neutrallight}\hrule width\headwidth height\footrulewidth}
}

% Style pour les pages de chapitre
\fancypagestyle{plain}{
    \fancyhf{}
    \fancyfoot[C]{\textcolor{neutraldark}{\thepage}}
    \renewcommand{\headrulewidth}{0pt}
    \renewcommand{\footrulewidth}{0pt}
}

% ===== PERSONNALISATION TABLE DES MATIÈRES =====

% Style du titre de la table des matières
% \renewcommand{\cfttoctitlefont}{\Large\bfseries\color{brandprimary}}

% Style des entrées de chapitre
\renewcommand{\cftchapfont}{\bfseries\color{neutraldark}}
\renewcommand{\cftchappagefont}{\bfseries\color{neutraldark}}
\renewcommand{\cftchapleader}{\cftdotfill{\cftsecdotsep}}

% Style des entrées de section
\renewcommand{\cftsecfont}{\color{neutraldark}}
\renewcommand{\cftsecpagefont}{\color{neutralmedium}}
       % Styles et mise en page
% =====================================================
% MÉTADONNÉES PERSONNALISABLES DU PROJET
% Version corrigée avec \renewcommand
% =====================================================

% =====================================================
% DÉFINITION DES VARIABLES AVEC \renewcommand
% =====================================================

% ÉTAPE 1: Informations de base du rapport
\renewcommand{\reporttitle}{Tutorial: Professional Report Template Features}
\renewcommand{\reportsubtitle}{Complete Guide to LaTeX Template Customization}
\renewcommand{\reporttype}{TUTORIAL REPORT}

% ÉTAPE 2: Informations de l'étudiant
\renewcommand{\reportauthor}{John Doe Student}
\renewcommand{\studentid}{ST2024001}
\renewcommand{\program}{Computer Science Engineering}
\renewcommand{\academicyear}{2024-2025}
\renewcommand{\internshipperiod}{January 20 - June 20, 2025}

% ÉTAPE 3: Institution académique
\renewcommand{\university}{University of Technology}
\renewcommand{\department}{Computer Science Department}
\renewcommand{\academicsupervisor}{Dr. Academic Supervisor}
\renewcommand{\supervisorposition}{Professor in Software Engineering}

% ÉTAPE 4: Entreprise d'accueil
\renewcommand{\company}{Tech Innovation Corp}
\renewcommand{\companydepartment}{Software Development Department}
\renewcommand{\companytutor}{Jane Smith}
\renewcommand{\tutorposition}{Senior Software Architect}

% ÉTAPE 5: Informations complémentaires
\renewcommand{\defensedate}{July 15, 2025}

% Variables personnalisées supplémentaires (optionnel)
\renewcommand{\projectname}{AgileFlow}
\renewcommand{\maintech}{React.js}
\renewcommand{\backendtech}{Node.js}
\renewcommand{\databasetech}{MongoDB}


% =====================================================
% COMMANDES DE VÉRIFICATION (pour debug)
% =====================================================

% Commande pour tester que les variables sont bien définies
\renewcommand{\testvariables}{%
    \textbf{TEST DES VARIABLES :}\\[0.5cm]
    \begin{tabular}{|l|l|}
    \hline
    Variable & Valeur \\
    \hline
    reporttitle & \reporttitle \\
    \hline
    reportauthor & \reportauthor \\
    \hline
    university & \university \\
    \hline
    company & \company \\
    \hline
    defensedate & \defensedate \\
    \hline
    \end{tabular}
}       % Styles et mise en page

% Templates réutilisables
% =====================================================
% ENVIRONNEMENTS PERSONNALISÉS
% =====================================================

% Boîte d'information
\newtcolorbox{infobox}[1][]{
    enhanced,
    colback=primarycolor!10,
    colframe=primarycolor,
    boxrule=1pt,
    arc=3pt,
    left=10pt,
    right=10pt,
    top=8pt,
    bottom=8pt,
    fonttitle=\bfseries\color{white},
    title={Information},
    #1
}

% Boîte d'avertissement
\newtcolorbox{warningbox}[1][]{
    enhanced,
    colback=orange!10,
    colframe=orange,
    boxrule=1pt,
    arc=3pt,
    left=10pt,
    right=10pt,
    top=8pt,
    bottom=8pt,
    fonttitle=\bfseries\color{white},
    title={Warning},
    #1
}

% Boîte de succès
\newtcolorbox{successbox}[1][]{
    enhanced,
    colback=secondarycolor!10,
    colframe=secondarycolor,
    boxrule=1pt,
    arc=3pt,
    left=10pt,
    right=10pt,
    top=8pt,
    bottom=8pt,
    fonttitle=\bfseries\color{white},
    title={Success},
    #1
}

% Environnement pour objectifs
% \newenvironment{objectives}
% {
%     \par\vspace{10pt}
%     \noindent\textbf{\color{primarycolor}Objectives:}
%     \par\vspace{5pt}
% }
% {
%     \par\vspace{5pt}
% }

% Environnement pour résultats
% \newenvironment{results}
% {
%     \par\vspace{10pt}
%     \noindent\textbf{\color{secondarycolor}Results:}
%     \par\vspace{5pt}
% }
% {
%     \par\vspace{5pt}
% }

% Environnement pour définitions
% \newenvironment{definition}[1]
% {
%     \par\vspace{10pt}
%     \noindent\textbf{\color{primarycolor}Definition - #1:}
%     \par\vspace{5pt}
%     \begin{quote}
% }
% {
%     \end{quote}
%     \par\vspace{5pt}
% }

% Missing environments
\newtcolorbox{technicalbox}[1][]{
    enhanced,
    colback=blue!5,
    colframe=blue!50!black,
    boxrule=1pt,
    arc=3pt,
    left=10pt,
    right=10pt,
    top=8pt,
    bottom=8pt,
    fonttitle=\bfseries\color{white},
    title={Technical Information},
    #1
}

\newtcolorbox{bestpractice}[1][]{
    enhanced,
    colback=green!5,
    colframe=green!50!black,
    boxrule=1pt,
    arc=3pt,
    left=10pt,
    right=10pt,
    top=8pt,
    bottom=8pt,
    fonttitle=\bfseries\color{white},
    title={Best Practice},
    #1
}

\newtcolorbox{lessonslearned}[1][]{
    enhanced,
    colback=purple!5,
    colframe=purple!50!black,
    boxrule=1pt,
    arc=3pt,
    left=10pt,
    right=10pt,
    top=8pt,
    bottom=8pt,
    fonttitle=\bfseries\color{white},
    title={Lessons Learned},
    #1
}
    % Boîtes et encadrés
% =====================================================
% TEMPLATES DE FIGURES
% =====================================================

% Template de figure standard (légende en bas)
\newcommand{\standardfigure}[4]{
  \begin{figure}[H]
    \centering
    \includegraphics[width=#2, keepaspectratio]{#1}
    \caption{#3}
    \label{#4}
  \end{figure}
}

% Template de figure avec cadre
\newcommand{\framedfigure}[4]{
  \begin{figure}[H]
    \centering
    \fbox{\includegraphics[width=#2, keepaspectratio]{#1}}
    \caption{#3}
    \label{#4}
  \end{figure}
}

% Template de figure simple sans cadre
\newcommand{\simplefigure}[4]{
  \begin{figure}[h]
    \centering
    \includegraphics[width=#2]{#1}
    \caption{#3}
    \label{#4}
  \end{figure}
}

% Template pour diagrammes avec bordure colorée
\newcommand{\diagramfigure}[5]{
  \begin{figure}[H]
    \centering
    \begin{tcolorbox}[colback=white,colframe=primarycolor,width=#2]
      \centering
      \includegraphics[width=0.9\textwidth, height=#3, keepaspectratio]{#1}
    \end{tcolorbox}
    \caption{#4}
    \label{#5}
  \end{figure}
}

% Template pour figures côte à côte
\newcommand{\doublefigure}[8]{
  \begin{figure}[H]
    \centering
    \begin{subfigure}[b]{#3}
      \centering
      \includegraphics[width=\textwidth]{#1}
      \caption{#4}
      \label{#5}
    \end{subfigure}
    \hfill
    \begin{subfigure}[b]{#6}
      \centering
      \includegraphics[width=\textwidth]{#2}
      \caption{#7}
      \label{#8}
    \end{subfigure}
    \caption{Comparison figures}
  \end{figure}
}
  % Templates de figures
% =====================================================
% TEMPLATES DE TABLEAUX
% =====================================================

% Template de tableau standard
\newenvironment{standardtable}[2]
{
  \begin{table}[h]
  \centering
  \rowcolors{2}{tablealternate}{white}
  \begin{tabular}{#1}
  \rowcolor{tableheader}
  #2 \\
  \hline
}
{
  \hline
  \end{tabular}
  \end{table}
}

% Template de tableau avec alternance
\newenvironment{alternatingtable}[2]
{
  \begin{table}[h]
  \centering
  \rowcolors{2}{tablealternate}{white}
  \begin{tabular}{#1}
  \rowcolor{tableheader}
  #2 \\
  \hline
}
{
  \hline
  \end{tabular}
  \end{table}
}

% Template de tableau coloré
\newenvironment{coloredtable}[3]
{
  \begin{table}[h]
  \centering
  \rowcolors{2}{#3}{white}
  \begin{tabular}{#1}
  \rowcolor{tableheader}
  #2 \\
  \hline
}
{
  \hline
  \end{tabular}
  \end{table}
}

% Template pour tableau de métriques
\newcommand{\metricstable}[1]{
  \begin{table}[h]
  \centering
  \begin{tabular}{|>{\columncolor{tableheader}}p{5cm}|>{\centering\arraybackslash}p{2cm}|>{\centering\arraybackslash}p{2cm}|>{\centering\arraybackslash\color{textcolor}\bfseries}p{2cm}|}
  \hline
  \rowcolor{tableheader}
  \textbf{Indicator} & \textbf{Before} & \textbf{After} & \textbf{Improvement} \\
  \hline
  #1
  \end{tabular}
  \end{table}
}

% Template de tableau simple sans bordures
\newenvironment{simpletable}[2]
{
  \begin{table}[h]
  \centering
  \begin{tabular}{#1}
  \toprule
  #2 \\
  \midrule
}
{
  \bottomrule
  \end{tabular}
  \end{table}
}
   % Templates de tableaux

% =====================================================
% DÉBUT DU DOCUMENT
% =====================================================

\begin{document}

% =====================================================
% PAGE DE GARDE AUTOMATIQUE
% =====================================================

% \makereportpage

% =====================================================
% PAGE DE GARDE INTÉGRÉE - COMPATIBLE AVEC LE PROJET
% =====================================================

% Commandes personnalisées pour les encadrants
\newcommand{\academicsupervisorinfo}[2]{%
    \textbf{Academic Supervisor}\\[0.1cm]
    \textcolor{primarycolor}{\textbf{#1}}\\[0.1cm]
    \textcolor{graycolor}{\textit{#2}}\\[0.1cm]
}

\newcommand{\companysupervisorinfo}[2]{%
    \textbf{Company Supervisor}\\[0.1cm]
    \textcolor{primarycolor}{\textbf{#1}}\\[0.1cm]
    \textcolor{graycolor}{\textit{#2}}\\[0.1cm]
}

\newcommand{\documentinfo}[1]{%
    \textcolor{textcolor}{\textbf{Academic Year: #1}}\\[0.5cm]
    \textcolor{graycolor}{\textit{Generated on \today}}
}

\begin{titlepage}
    % Réduction temporaire des marges pour cette page uniquement
    \newgeometry{top=1.27cm,bottom=1.27cm,left=1.27cm,right=1.27cm}
    \thispagestyle{empty}

    % Désactive l'interligne 1,5 pour la page de garde
    {\singlespacing

    % ===== BORDURE SIMPLE ET ÉLÉGANTE =====
    \begin{tikzpicture}[remember picture,overlay]
        % Variables pour faciliter les ajustements
        \def\bordermarg{1cm}
        \def\innerspace{0.3cm}
        
        % Bordure principale double
        \draw[primarycolor, line width=2pt] 
            ([xshift=\bordermarg,yshift=-\bordermarg]current page.north west) 
            rectangle 
            ([xshift=-\bordermarg,yshift=\bordermarg]current page.south east);
        
        \draw[primarycolor, line width=0.8pt] 
            ([xshift=\bordermarg+\innerspace,yshift=-\bordermarg-\innerspace]current page.north west) 
            rectangle 
            ([xshift=-\bordermarg-\innerspace,yshift=\bordermarg+\innerspace]current page.south east);
        
        % Ornements des coins simplifiés
        % Coin supérieur gauche
        \draw[primarycolor, line width=1.5pt] 
            ([xshift=\bordermarg,yshift=-\bordermarg]current page.north west) 
            -- ++(1.2,0) -- ++(0,-0.4) -- ++(-0.4,0) -- ++(0,-1.2) -- ++(-0.8,0) -- cycle;
        \draw[primarycolor, line width=1pt, rotate around={45:([xshift=\bordermarg+0.6cm,yshift=-\bordermarg-0.6cm]current page.north west)}] 
            ([xshift=\bordermarg+0.6cm,yshift=-\bordermarg-0.6cm]current page.north west) rectangle ++(0.15,0.15);
        
        % Coin supérieur droit
        \draw[primarycolor, line width=1.5pt] 
            ([xshift=-\bordermarg,yshift=-\bordermarg]current page.north east) 
            -- ++(-1.2,0) -- ++(0,-0.4) -- ++(0.4,0) -- ++(0,-1.2) -- ++(0.8,0) -- cycle;
        \draw[primarycolor, line width=1pt, rotate around={45:([xshift=-\bordermarg-0.6cm,yshift=-\bordermarg-0.6cm]current page.north east)}] 
            ([xshift=-\bordermarg-0.6cm,yshift=-\bordermarg-0.6cm]current page.north east) rectangle ++(0.15,0.15);
        
        % Coin inférieur gauche
        \draw[primarycolor, line width=1.5pt] 
            ([xshift=\bordermarg,yshift=\bordermarg]current page.south west) 
            -- ++(1.2,0) -- ++(0,0.4) -- ++(-0.4,0) -- ++(0,1.2) -- ++(-0.8,0) -- cycle;
        \draw[primarycolor, line width=1pt, rotate around={45:([xshift=\bordermarg+0.6cm,yshift=\bordermarg+0.6cm]current page.south west)}] 
            ([xshift=\bordermarg+0.6cm,yshift=\bordermarg+0.6cm]current page.south west) rectangle ++(0.15,0.15);
        
        % Coin inférieur droit
        \draw[primarycolor, line width=1.5pt] 
            ([xshift=-\bordermarg,yshift=\bordermarg]current page.south east) 
            -- ++(-1.2,0) -- ++(0,0.4) -- ++(0.4,0) -- ++(0,1.2) -- ++(0.8,0) -- cycle;
        \draw[primarycolor, line width=1pt, rotate around={45:([xshift=-\bordermarg-0.6cm,yshift=\bordermarg+0.6cm]current page.south east)}] 
            ([xshift=-\bordermarg-0.6cm,yshift=\bordermarg+0.6cm]current page.south east) rectangle ++(0.15,0.15);
    \end{tikzpicture}

    % Contenu centré avec espacement amélioré
    \begin{center}
        % Logos en haut avec espacement (optionnels - remplacez par vos logos)
        % \vspace{0.5cm}
        \begin{minipage}{0.35\textwidth}
            \centering
            \includegraphics[width=0.6\textwidth]{assets/logos/university-logo.png}
            % {\Large \textcolor{primarycolor}{\textbf{[UNIVERSITY LOGO]}}}
        \end{minipage}%
        \hfill
        \begin{minipage}{0.35\textwidth}
            \centering
            \includegraphics[width=0.6\textwidth]{assets/logos/company-logo.png}
            % {\Large \textcolor{secondarycolor}{\textbf{[COMPANY LOGO]}}}
        \end{minipage}

        \vspace{1cm}

        % Section académique
        % {\large \textcolor{textcolor}{\textbf{\university}}}\\[0.3cm]
        % {\large \textcolor{primarycolor}{\textbf{\department}}}\\[0.3cm]
        % {\large \textcolor{graycolor}{\program}}\\[1cm]

        % Type de document
        {\huge \textcolor{textcolor}{\textbf{\reporttype}}}\\[0.75cm]
        \textcolor{primarycolor}{\rule{0.8\linewidth}{0.5mm}}\\[0.5cm]

        % Titre principal
        {\normalsize
        \begin{minipage}{0.85\linewidth}
            \centering
            {\LARGE \textbf{\textcolor{primarycolor}{\reporttitle{}}}}
        \end{minipage}\\[0.5cm]
        
        % Sous-titre
        \begin{minipage}{0.8\linewidth}
            \centering
            {\large \textcolor{secondarycolor}{\reportsubtitle}}
        \end{minipage}\\[0.5cm]
        
        \textcolor{textcolor}{\rule{0.6\linewidth}{0.5mm}}\\[0.5cm]
        }

        % Contexte du projet
        {\large \textcolor{textcolor}{Project conducted at \textcolor{primarycolor}{\company}}}\\[0.2cm]
        {\large \textcolor{graycolor}{Internship Period: \internshipperiod}}\\[1cm]

        % Auteur mis en emphase
        {\large \textcolor{textcolor}{Prepared by}}\\[0.3cm]
        {\large \textcolor{primarycolor}{\textbf{\reportauthor}}}\\[0.25cm]
        {\large \textcolor{graycolor}{Student ID: \studentid}}\\[0.25cm]
        {\large \textcolor{graycolor}{\program}}\\[1cm]
    \end{center}

    \vspace{1cm}

    % Encadrants avec nouvelle mise en page centrée
    \begin{center}
        \begin{minipage}{0.42\textwidth}
            \centering
            \companysupervisorinfo{\companytutor}{\tutorposition}
        \end{minipage}
        \hfill
        \begin{minipage}{0.42\textwidth}
            \centering
            \academicsupervisorinfo{\academicsupervisor}{\supervisorposition}
        \end{minipage}%
    \end{center}

    \vspace{1cm}

    % Informations académiques centrées
    \begin{center}
        \documentinfo{\academicyear}\\[0.3cm]
        {\large \textcolor{primarycolor}{\textbf{Defense Date: \defensedate}}}
    \end{center}

    \vspace{1cm}

    } % Fin du bloc singlespacing

    % Restauration des marges originales à la fin de la page
    \restoregeometry
\end{titlepage}


% =====================================================
% PAGES LIMINAIRES (FRONT MATTER)
% =====================================================
\newgeometry{top=1.75cm,bottom=1.75cm,left=2.54cm,right=2.54cm}

\frontmatter

% =====================================================
% SOMMAIRES AUTOMATIQUES
% =====================================================

% % Table des matières
% \tableofcontents
% \newpage

% % Liste des figures (générée automatiquement)
% \listoffigures
% \newpage

% % Liste des tableaux (générée automatiquement)
% \listoftables
% \newpage

% Table des matières
\newpage
\chapter*{Contents}
\vspace*{-2cm} % Ajustez la valeur pour réduire l'espace si besoin
\addcontentsline{toc}{chapter}{Contents}
\begin{singlespace}
\renewcommand{\contentsname}{}
\vspace*{-1cm}
\tableofcontents
\end{singlespace}

% Liste des tableaux
\newpage
\chapter*{List of Tables}
\vspace*{-2cm} % Ajustez la valeur pour réduire l'espace si besoin
\addcontentsline{toc}{chapter}{List of Tables}
\begin{singlespace}
\renewcommand{\listtablename}{}
\vspace*{-1cm}
\listoftables
\end{singlespace}

% Liste des figures
\newpage
\chapter*{List of Figures}
\vspace*{-2cm} % Ajustez la valeur pour réduire l'espace si besoin
\addcontentsline{toc}{chapter}{List of Figures}
\begin{singlespace}
\renewcommand{\listfigurename}{}
\vspace*{-1cm} % Ajustez la valeur pour réduire l'espace si besoin
\listoffigures
\end{singlespace}


% =====================================================
% PAGES LIMINAIRES PERSONNALISÉES
% =====================================================

\chapter*{List of Abbreviations}
\addcontentsline{toc}{chapter}{List of Abbreviations}

\begin{table}[H]
\centering
\begin{tabular}{|p{3cm}|p{10cm}|}
\hline
\tableheadercell{Abbreviation} & \tableheadercell{Definition} \\
\hline
API & Application Programming Interface \\
\hline
\alternaterow
CRUD & Create, Read, Update, Delete \\
\hline
HTTP & HyperText Transfer Protocol \\
\hline
\alternaterow
JSON & JavaScript Object Notation \\
\hline
REST & Representational State Transfer \\
\hline
\alternaterow
SQL & Structured Query Language \\
\hline
UI/UX & User Interface/User Experience \\
\hline
\end{tabular}
\caption{List of Technical Abbreviations}
\end{table}

\chapter*{Abstract}
\addcontentsline{toc}{chapter}{Abstract}

\section*{English Abstract}

This tutorial document demonstrates the comprehensive features of the professional internship report LaTeX template. It showcases advanced typography, automated content generation, custom environments, and professional styling options.

\begin{objectives}
The tutorial objectives include:
\begin{itemize}
\item Demonstrating automatic table of contents generation
\item Showcasing figure and table insertion methods
\item Illustrating custom commands and environments
\item Presenting typography and styling options
\end{itemize}
\end{objectives}

\begin{results}
Key features demonstrated:
\begin{itemize}
\item \metric{15+} custom environments and commands
\item \metric{100\%} automated content indexing
\item \metric{Professional} typography and layout
\item \metric{Cross-platform} compatibility
\end{itemize}
\end{results}

\textbf{Keywords:} LaTeX, template, tutorial, typography, automation, professional

\chapter*{Acknowledgments}
\addcontentsline{toc}{chapter}{Acknowledgments}

I would like to thank the open-source LaTeX community for providing the tools and packages that make professional document creation possible. Special thanks to the developers of \technology{tcolorbox}, \technology{tikz}, and \technology{hyperref} packages.

% =====================================================
% CORPS PRINCIPAL (MAIN MATTER)
% =====================================================

\mainmatter

\chapter{Introduction: Template Features Overview}
\label{chap:introduction}

\section{Welcome to the Professional Template}

This tutorial demonstrates all features of the internship report template. Each section showcases different capabilities, from basic formatting to advanced automation features.

\begin{infobox}[Template Philosophy]
This template is designed with three core principles:
\begin{itemize}
\item \important{Modularity}: Separate configuration for easy customization
\item \important{Automation}: Reduce manual work with smart automation
\item \important{Professionalism}: Industry-standard appearance and typography
\end{itemize}
\end{infobox}

\section{Document Structure}

The template follows a hierarchical structure:

\begin{enumerate}
\item \textbf{Configuration Layer}: Modular settings in \filename{config/} directory
\item \textbf{Content Layer}: Structured content in \filename{content/} directory
\item \textbf{Template Layer}: Reusable components in \filename{templates/} directory
\item \textbf{Asset Layer}: Images and resources in \filename{assets/} directory
\end{enumerate}

\chapter{Typography and Text Formatting}
\label{chap:typography}

\section{Font Configuration}

The template uses \important{Times New Roman} as the primary font family, providing a professional appearance suitable for academic and corporate documents.

\subsection{Text Emphasis Commands}

Demonstrate various text emphasis options:

\begin{itemize}
\item \textbf{Bold text} for emphasis
\item \textit{Italic text} for concepts
\item \texttt{Monospace text} for code
\item \important{Important text} in brand color
\item \highlight{Highlighted text} for key points
\item \concept{Conceptual text} for definitions
\end{itemize}

\subsection{Specialized Text Commands}

Technical and domain-specific commands:

\begin{itemize}
\item Company: \companyref{Tech Innovation Corp}
\item Technology: \technology{React.js}, \technology{Node.js}
\item Programming: \programming{JavaScript}, \programming{Python}
\item Framework: \framework{Spring Boot}, \framework{Angular}
\item Database: \database{PostgreSQL}, \database{MongoDB}
\item Metrics: \metric{95\% uptime}, \improvement{40\% faster}
\end{itemize}

\section{Lists and Enumerations}

\subsection{Bullet Points}

Professional bullet point styling:

\begin{itemize}
\item First level items use colored bullets
\item Second level items:
    \begin{itemize}
    \item Use different styling
    \item Maintain visual hierarchy
    \item Support deep nesting
    \end{itemize}
\item Third level and beyond maintain consistency
\end{itemize}

\subsection{Numbered Lists}

Automatic numbering with professional spacing:

\begin{enumerate}
\item Requirements analysis and documentation
\item System design and architecture planning
\item Implementation and development phase
\item Testing and quality assurance
\item Deployment and maintenance
\end{enumerate}

\section{Paragraph Spacing and Line Height}

The template uses \texttt{onehalfspacing} (1.5x line height) for optimal readability. Paragraph spacing is set to \texttt{6pt} to create clean separation between paragraphs while maintaining flow.

\chapter{Custom Environments and Boxes}
\label{chap:environments}

\section{Information Boxes}

Various types of information boxes for different purposes:

\begin{infobox}[Key Information]
This is an information box used to highlight important details that readers should notice.
\end{infobox}

\begin{warningbox}
Warning boxes draw attention to potential issues or important considerations.
\end{warningbox}

\begin{successbox}
Success boxes highlight positive outcomes, achievements, or completed milestones.
\end{successbox}

\begin{technicalbox}
Technical boxes provide detailed technical information, specifications, or implementation details.
\end{technicalbox}

\section{Academic Environments}

\begin{definition}{Microservices Architecture}
A software development technique that structures an application as a collection of loosely coupled services, which implement business capabilities.
\end{definition}

\begin{objectives}
Project objectives for this tutorial:
\begin{itemize}
\item Demonstrate all template features
\item Provide practical examples
\item Show best practices
\item Enable quick adoption
\end{itemize}
\end{objectives}

\begin{results}
Tutorial outcomes:
\begin{itemize}
\item \metric{100\%} feature coverage
\item \improvement{Reduced learning time}
\item \highlight{Professional output quality}
\end{itemize}
\end{results}

\begin{keyfindings}
Key insights from template development:
\begin{itemize}
\item Modular architecture improves maintainability
\item Automation reduces manual errors
\item Professional styling enhances credibility
\end{itemize}
\end{keyfindings}

\section{Project Management Environments}

\begin{methodology}
The template development followed \methodology{Agile} principles with iterative improvements and user feedback integration.
\end{methodology}

User story example:
\userstory{template user}{to quickly generate professional reports}{I can focus on content rather than formatting}

\chapter{Tables and Data Presentation}
\label{chap:tables}

\section{Basic Table Formatting}

Simple table with alternating row colors:

\begin{table}[H]
\centering
\caption{Basic Table Example with Professional Styling}
\label{tab:basic-example}
\begin{tabular}{|p{4cm}|p{6cm}|p{3cm}|}
\hline
\tableheadercell{Feature} & \tableheadercell{Description} & \tableheadercell{Status} \\
\hline
Automated TOC & Generates table of contents automatically & \important{Active} \\
\hline
\alternaterow
Custom Colors & Professional color scheme & \important{Active} \\
\hline
Responsive Layout & Adapts to different page sizes & \important{Active} \\
\hline
\alternaterow
Cross-references & Automatic figure and table numbering & \important{Active} \\
\hline
\end{tabular}
\end{table}

\section{Performance Metrics Table}

Specialized table for showing performance improvements:

\begin{table}[H]
\centering
\caption{Performance Improvement Metrics}
\label{tab:performance-metrics}
\begin{tabular}{|p{5cm}|p{2.5cm}|p{2.5cm}|p{2.5cm}|}
\hline
\tableheadercell{Metric} & \tableheadercell{Before} & \tableheadercell{After} & \tableheadercell{Improvement} \\
\hline
Page Load Time & 3.2s & 1.8s & \improvement{44\%} \\
\hline
\alternaterow
Memory Usage & 512MB & 320MB & \improvement{38\%} \\
\hline
Error Rate & 2.1\% & 0.3\% & \improvement{86\%} \\
\hline
\alternaterow
User Satisfaction & 72\% & 94\% & \improvement{31\%} \\
\hline
\end{tabular}
\end{table}

\section{Complex Data Table}

Table with multiple data types and formatting:

\begin{table}[H]
\centering
\caption{Technology Stack Comparison}
\label{tab:tech-stack}
\begin{tabular}{|p{3cm}|p{3cm}|p{3cm}|p{4cm}|}
\hline
\tableheadercell{Category} & \tableheadercell{Technology} & \tableheadercell{Version} & \tableheadercell{Use Case} \\
\hline
Frontend & \framework{React} & 18.2.0 & User interface development \\
\hline
\alternaterow
Backend & \framework{Node.js} & 18.17.0 & Server-side logic \\
\hline
Database & \database{PostgreSQL} & 15.3 & Data persistence \\
\hline
\alternaterow
Cache & \database{Redis} & 7.0.11 & Session management \\
\hline
Deployment & \technology{Docker} & 24.0.2 & Containerization \\
\hline
\end{tabular}
\end{table}

\chapter{Figures and Image Management}
\label{chap:figures}

\section{Standard Figure Insertion}

Basic figure with caption and cross-reference:

\begin{figure}[H]
\centering
\includegraphics[width=0.8\textwidth]{assets/images/examples/architecture-diagram.png}
\caption{System Architecture Overview - demonstrates automatic figure numbering}
\label{fig:architecture}
\end{figure}

Reference to the figure: As shown in \figref{architecture}, the system follows a microservices architecture pattern.

\section{Multiple Figure Layouts}

\subsection{Side-by-side Figures}

\begin{figure}[H]
\centering
\begin{subfigure}[b]{0.45\textwidth}
\includegraphics[width=\textwidth]{assets/images/examples/before-optimization.png}
\caption{Before optimization}
\label{fig:before}
\end{subfigure}
\hfill
\begin{subfigure}[b]{0.45\textwidth}
\includegraphics[width=\textwidth]{assets/images/examples/after-optimization.png}
\caption{After optimization}
\label{fig:after}
\end{subfigure}
\caption{Performance Comparison: Before and After Optimization}
\label{fig:comparison}
\end{figure}

\subsection{Wrapped Figure with Text}

\begin{wrapfigure}{r}{0.4\textwidth}
\centering
\includegraphics[width=0.38\textwidth]{assets/images/examples/mobile-interface.png}
\caption{Mobile interface}
\label{fig:mobile}
\end{wrapfigure}

This text wraps around the figure, demonstrating how to integrate images with flowing text content. The mobile interface shown provides an intuitive user experience optimized for touch interactions.

The responsive design ensures optimal display across different screen sizes and orientations. Key features include simplified navigation, touch-friendly buttons, and streamlined information hierarchy.

\section{Technical Diagrams}

\begin{figure}[H]
\centering
\includegraphics[width=0.9\textwidth]{assets/images/diagrams/database-schema.png}
\caption{Database Schema Design with Entity Relationships}
\label{fig:database-schema}
\end{figure}

\section{Charts and Graphs}

\begin{figure}[H]
\centering
\includegraphics[width=0.8\textwidth]{assets/images/charts/performance-trends.png}
\caption{Performance Trends Over Time - showing system improvements}
\label{fig:performance-trends}
\end{figure}

\chapter{Code Listings and Technical Content}
\label{chap:code}

\section{Code Formatting}

\subsection{Inline Code}

Use \programming{console.log()} for debugging or \tech{npm install} for package installation.

\subsection{Code Blocks}

\begin{lstlisting}[language=JavaScript, caption=React Component Example]
import React, { useState, useEffect } from 'react';

const UserDashboard = ({ userId }) => {
    const [userData, setUserData] = useState(null);
    const [loading, setLoading] = useState(true);

    useEffect(() => {
        fetchUserData(userId)
            .then(data => {
                setUserData(data);
                setLoading(false);
            })
            .catch(error => {
                console.error('Error fetching user data:', error);
                setLoading(false);
            });
    }, [userId]);

    if (loading) {
        return <div>Loading...</div>;
    }

    return (
        <div className="dashboard">
            <h1>Welcome, {userData.name}</h1>
            <div className="stats">
                <p>Projects: {userData.projectCount}</p>
                <p>Last Login: {userData.lastLogin}</p>
            </div>
        </div>
    );
};

export default UserDashboard;
\end{lstlisting}

\subsection{Multiple Language Examples}

Python example:

\begin{lstlisting}[language=Python, caption=Data Processing Script]
import pandas as pd
import numpy as np
from sklearn.model_selection import train_test_split

def process_data(file_path):
    """
    Process raw data and prepare for machine learning
    """
    # Load and clean data
    df = pd.read_csv(file_path)
    df = df.dropna()
    
    # Feature engineering
    df['feature_ratio'] = df['feature_a'] / df['feature_b']
    df['log_transform'] = np.log1p(df['target_variable'])
    
    # Split data
    X = df.drop('target_variable', axis=1)
    y = df['target_variable']
    
    return train_test_split(X, y, test_size=0.2, random_state=42)

# Usage
X_train, X_test, y_train, y_test = process_data('data.csv')
print(f"Training set size: {len(X_train)}")
print(f"Test set size: {len(X_test)}")
\end{lstlisting}

SQL example:

\begin{lstlisting}[language=SQL, caption=Database Query Optimization]
-- Optimized query with proper indexing
SELECT 
    u.user_id,
    u.username,
    COUNT(p.project_id) as project_count,
    AVG(p.completion_rate) as avg_completion
FROM users u
LEFT JOIN projects p ON u.user_id = p.owner_id
WHERE u.status = 'active'
    AND u.created_date >= DATE_SUB(NOW(), INTERVAL 1 YEAR)
GROUP BY u.user_id, u.username
HAVING project_count > 0
ORDER BY avg_completion DESC
LIMIT 100;

-- Index recommendations for performance
CREATE INDEX idx_users_status_date ON users(status, created_date);
CREATE INDEX idx_projects_owner ON projects(owner_id);
\end{lstlisting}

\section{Code Explanation Boxes}

\begin{codeexplanation}
The React component above demonstrates:
\begin{itemize}
\item \important{State Management}: Using \programming{useState} for component state
\item \important{Side Effects}: Managing API calls with \programming{useEffect}
\item \important{Error Handling}: Proper error catching and user feedback
\item \important{Conditional Rendering}: Loading states and data display
\end{itemize}
\end{codeexplanation}

\chapter{Cross-References and Navigation}
\label{chap:references}

\section{Automatic Numbering}

The template automatically numbers and cross-references all elements:

\begin{itemize}
\item Chapters: \chapref{introduction}, \chapref{typography}, \chapref{environments}
\item Sections: \secref{font-configuration}, \secref{lists-and-enumerations}
\item Figures: \figref{architecture}, \figref{comparison}, \figref{mobile}
\item Tables: \tabref{basic-example}, \tabref{performance-metrics}, \tabref{tech-stack}
\item Code listings: Listing~\ref{code:react-component}, Listing~\ref{code:python-script}
\end{itemize}

\section{Hyperlink Navigation}

All cross-references are clickable hyperlinks in the PDF, providing easy navigation throughout the document. The template automatically handles:

\begin{itemize}
\item \important{Internal Links}: All \textbackslash ref commands become clickable
\item \important{Table of Contents}: Clickable entries to jump to sections
\item \important{List of Figures/Tables}: Direct navigation to figures and tables
\item \important{Bibliography}: Linked citations (when bibliography is used)
\end{itemize}

\chapter{Advanced Features and Customization}
\label{chap:advanced}

\section{Color Scheme Customization}

The template uses a professional color scheme that can be easily customized in \filename{config/colors.tex}:

\begin{technicalbox}[Color Variables]
Primary colors:
\begin{itemize}
\item \textcolor{brandprimary}{\textbf{brandprimary}}: Main accent color (RGB: 220,80,80)
\item \textcolor{brandsecondary}{\textbf{brandsecondary}}: Secondary color (RGB: 0,82,147)
\item \textcolor{brandaccent}{\textbf{brandaccent}}: Accent color (RGB: 0,130,62)
\end{itemize}
\end{technicalbox}

\section{Typography Configuration}

Font and spacing settings can be modified in \filename{config/style.tex}:

\begin{itemize}
\item \textbf{Font Family}: Times New Roman (professional standard)
\item \textbf{Line Spacing}: 1.5x (\texttt{onehalfspacing})
\item \textbf{Paragraph Spacing}: 6pt with hanging indent
\item \textbf{Section Spacing}: Optimized for readability
\end{itemize}

\section{Template Structure}

\begin{bestpractice}
For optimal results when using this template:
\begin{enumerate}
\item Keep metadata in \filename{config/metadata.tex}
\item Store content in structured \filename{content/} directories
\item Place reusable elements in \filename{templates/}
\item Organize assets in \filename{assets/} with subdirectories
\item Use provided commands for consistency
\end{enumerate}
\end{bestpractice}

\section{Compilation Options}

The template supports multiple compilation methods:

\begin{table}[H]
\centering
\caption{Compilation Methods and Use Cases}
\label{tab:compilation-methods}
\begin{tabular}{|p{4cm}|p{5cm}|p{4cm}|}
\hline
\tableheadercell{Method} & \tableheadercell{Description} & \tableheadercell{Best For} \\
\hline
Full Compilation & Complete build with bibliography & Final documents \\
\hline
\alternaterow
Quick Compilation & Single pass compilation & Draft writing \\
\hline
Watch Mode & Automatic compilation on save & Active development \\
\hline
\alternaterow
Clean Build & Fresh compilation from scratch & Troubleshooting \\
\hline
\end{tabular}
\end{table}

\chapter{Best Practices and Tips}
\label{chap:best-practices}

\section{Content Organization}

\begin{objectives}
Follow these organizational principles:
\begin{itemize}
\item \important{One chapter per file}: Easier version control and collaboration
\item \important{Logical file naming}: Use descriptive, consistent names
\item \important{Asset organization}: Group related images in subdirectories
\item \important{Version control}: Track changes with Git or similar systems
\end{itemize}
\end{objectives}

\section{Writing Guidelines}

\subsection{Professional Writing Style}

\begin{itemize}
\item Use \important{active voice} whenever possible
\item Write \important{clear, concise sentences}
\item Maintain \important{consistent terminology}
\item Include \important{specific metrics} and quantified results
\end{itemize}

\subsection{Technical Documentation}

\begin{lessonslearned}
Key lessons for technical writing:
\begin{itemize}
\item \important{Explain before showing}: Provide context before code examples
\item \important{Use visual aids}: Diagrams clarify complex concepts
\item \important{Include examples}: Real-world examples aid understanding
\item \important{Maintain consistency}: Use the same terms throughout
\end{itemize}
\end{lessonslearned}

\section{Quality Assurance}

\subsection{Content Review Checklist}

\begin{enumerate}
\item \textbf{Spelling and Grammar}: Use spell-check and proof-reading
\item \textbf{Cross-references}: Verify all \textbackslash ref commands work
\item \textbf{Figure Quality}: Ensure images are high-resolution
\item \textbf{Table Formatting}: Check alignment and readability
\item \textbf{Code Accuracy}: Test all code examples
\item \textbf{Consistent Style}: Follow template conventions
\end{enumerate}

\subsection{Final Compilation Check}

Before submission, perform these checks:

\begin{warningbox}
Critical final checks:
\begin{itemize}
\item Compile successfully without errors
\item All figures display correctly
\item Table of contents is accurate
\item Page numbers are sequential
\item Bibliography is properly formatted
\item PDF bookmarks work correctly
\end{itemize}
\end{warningbox}

% =====================================================
% CONCLUSION
% =====================================================

\chapter*{Conclusion}
\addcontentsline{toc}{chapter}{Conclusion}

This tutorial has demonstrated the comprehensive features of the professional internship report template. The modular architecture, automated content generation, and professional styling provide a robust foundation for creating high-quality academic and professional documents.\cite{cohn2009succeeding}

\begin{keyfindings}
Tutorial outcomes:
\begin{itemize}
\item \metric{Complete coverage} of all template features
\item \improvement{Practical examples} for immediate implementation
\item \highlight{Professional standards} for document quality
\item \important{Automation benefits} for efficiency
\end{itemize}
\end{keyfindings}

The template's design philosophy of modularity, automation, and professionalism ensures that users can focus on content creation while maintaining consistent, high-quality output.

% =====================================================
% ANNEXES
% =====================================================

\appendix

\chapter{Template File Structure}
\label{app:file-structure}

Complete directory structure of the template:

\begin{lstlisting}[language=bash, caption=Template Directory Structure]
internship-report-template/
|-- .vscode/                    # VS Code configuration
|   |-- extensions.json
|   |-- settings.json
|   \-- tasks.json
|-- assets/                     # Project assets
|   |-- fonts/                 # Custom fonts
|   |-- images/                # Image files
|   |   |-- architecture/      # Architecture diagrams
|   |   |-- charts/           # Charts and graphs
|   |   |-- diagrams/         # Technical diagrams
|   |   |-- screenshots/      # Application screenshots
|   |   \-- examples/         # Tutorial examples
|   \-- logos/                # Company/institution logos
|-- build/                     # Compilation output
|-- config/                    # Modular configuration
|   |-- colors.tex            # Color definitions
|   |-- commands.tex          # Custom commands
|   |-- metadata.tex          # Document metadata
|   |-- packages.tex          # LaTeX packages
|   \-- style.tex             # Typography and layout
|-- content/                   # Document content
|   |-- frontmatter/          # Front matter pages
|   |-- chapters/             # Main chapters
|   \-- backmatter/           # Conclusion and appendices
|-- diagrams/                 # Diagram source files
|-- scripts/                  # Build scripts
|   |-- compile.bat           # Windows compilation
|   |-- compile.sh            # Unix/Mac compilation
|   |-- clean.bat             # Windows cleanup
|   \-- clean.sh              # Unix/Mac cleanup
|-- templates/                # Reusable templates
|   |-- boxes.tex             # Custom environments
|   |-- figures.tex           # Figure templates
|   \-- tables.tex            # Table templates
|-- internshipreport.cls      # LaTeX class file
|-- main.tex                  # Main document
|-- README.md                 # Documentation
\-- .gitignore               # Git ignore rules
\end{lstlisting}

\chapter{Custom Commands Reference}
\label{app:commands}

\section{Text Formatting Commands}

\begin{table}[H]
\centering
\caption{Text Formatting Commands Reference}
\begin{tabular}{|p{5cm}|p{5cm}|p{3cm}|}
\hline
\tableheadercell{Command} & \tableheadercell{Description} & \tableheadercell{Example} \\
\hline
\texttt{\textbackslash important\{text\}} & Important emphasis & \important{Important} \\
\hline
\alternaterow
\texttt{\textbackslash highlight\{text\}} & Highlight text & \highlight{Highlighted} \\
\hline
\texttt{\textbackslash concept\{text\}} & Conceptual term & \concept{Concept} \\
\hline
\alternaterow
\texttt{\textbackslash keyword\{text\}} & Keyword emphasis & \keyword{Keyword} \\
\hline
\end{tabular}
\end{table}

\section{Technical Commands}

\begin{table}[H]
\centering
\caption{Technical Commands Reference}
\begin{tabular}{|p{5cm}|p{5cm}|p{3cm}|}
\hline
\tableheadercell{Command} & \tableheadercell{Description} & \tableheadercell{Example} \\
\hline
\texttt{\textbackslash technology\{name\}} & Technology reference & \technology{React} \\
\hline
\alternaterow
\texttt{\textbackslash programming\{lang\}} & Programming language & \programming{Python} \\
\hline
\texttt{\textbackslash framework\{name\}} & Framework reference & \framework{Django} \\
\hline
\alternaterow
\texttt{\textbackslash database\{name\}} & Database reference & \database{PostgreSQL} \\
\hline
\texttt{\textbackslash companyref\{name\}} & Company reference & \companyref{Microsoft} \\
\hline
\end{tabular}
\end{table}

\chapter{Environment Usage Guide}
\label{app:environments}

\section{Information Environments}

Demonstration of all available box environments:

\begin{infobox}[Information Box]
Use for general information that enhances understanding.
\end{infobox}

\begin{warningbox}
Use for warnings, cautions, or important considerations.
\end{warningbox}

\begin{successbox}
Use for positive outcomes, achievements, or successful implementations.
\end{successbox}

\begin{technicalbox}
Use for detailed technical information, specifications, or implementation details.
\end{technicalbox}

\begin{bestpractice}
Use for recommended practices, guidelines, or proven methods.
\end{bestpractice}

\begin{lessonslearned}
Use for insights gained from experience, retrospectives, or project conclusions.
\end{lessonslearned}

\section{Academic Environments}

\begin{definition}{Template}
A pre-designed document format that provides structure and styling for consistent document creation.
\end{definition}

\begin{objectives}
Use for listing project objectives, goals, or intended outcomes.
\end{objectives}

\begin{results}
Use for presenting quantified results, metrics, or achievements.
\end{results}

\begin{keyfindings}
Use for summarizing important discoveries or conclusions.
\end{keyfindings}

\begin{methodology}
Use for describing approaches, methods, or procedures followed.
\end{methodology}

\chapter*{Bibliographie}
\nocite{*}
\addcontentsline{toc}{chapter}{Bibliographie}
\printbibliography[heading=none]

\end{document}