% =====================================================
% COMMANDES PERSONNALISÉES POUR LE TUTORIEL
% =====================================================

% ===== COMMANDES SPÉCIFIQUES AU PROJET =====

% Références d'entreprise et organisations
\newcommand{\companyref}[1]{\textcolor{primarycolor}{\textbf{#1}}}
\newcommand{\universityref}[1]{\textcolor{secondarycolor}{\textbf{#1}}}

% Nom du projet et technologies principales
\newcommand{\projectref}{\textbf{\color{brandprimary}\projectname}}
\newcommand{\mainframework}{\framework{\maintech}}
\newcommand{\backend}{\framework{\backendtech}}
\newcommand{\maindatabase}{\database{\databasetech}}
\newcommand{\programming}[1]{\texttt{\color{primarycolor}#1}}

% ===== COMMANDES POUR LES MÉTHODOLOGIES =====

% Méthodologies de développement
\newcommand{\agile}{\methodology{Agile}}
\newcommand{\scrum}{\methodology{Scrum}}
\newcommand{\kanban}{\methodology{Kanban}}
\newcommand{\devops}{\methodology{DevOps}}

% Éléments Scrum spécifiques
\newcommand{\sprintduration}[1]{\duration{#1-week sprint}}
\newcommand{\userstoryref}[1]{\textbf{\color{brandaccent}US-#1}}
\newcommand{\taskref}[1]{\textbf{\color{brandsecondary}TASK-#1}}

% ===== COMMANDES POUR LES TESTS =====

% Types de tests
\newcommand{\unittest}{\testcase{Unit Test}}
\newcommand{\integrationtest}{\testcase{Integration Test}}
% \newcommand{\e2etest}{\testcase{End-to-End Test}}
\newcommand{\performancetest}{\testcase{Performance Test}}

% Couverture de tests
\newcommand{\codecoverage}[1]{\metric{#1\% coverage}}
\newcommand{\testpass}[1]{\textcolor{brandsuccess}{\textbf{#1 passed}}}
\newcommand{\testfail}[1]{\textcolor{branddanger}{\textbf{#1 failed}}}

% ===== COMMANDES POUR LES PERFORMANCES =====

% Métriques de performance
\newcommand{\loadtime}[1]{\metric{#1ms load time}}
\newcommand{\responsetime}[1]{\metric{#1ms response}}
\newcommand{\throughput}[1]{\metric{#1 req/sec}}
\newcommand{\concurrent}[1]{\metric{#1 concurrent users}}

% Améliorations de performance
\newcommand{\speedup}[1]{\improvement{#1× faster}}
\newcommand{\reduction}[1]{\improvement{#1\% reduction}}
\newcommand{\increase}[1]{\improvement{#1\% increase}}

% ===== COMMANDES POUR LA SÉCURITÉ =====

% Éléments de sécurité
\newcommand{\authentication}{\important{Authentication}}
\newcommand{\authorization}{\important{Authorization}}
\newcommand{\encryption}{\important{Encryption}}
\newcommand{\validation}{\important{Input Validation}}

% Types d'utilisateurs et rôles
% \newcommand{\userrole}[1]{\role{#1}}
\newcommand{\adminrole}{\role{Administrator}}
\newcommand{\userrole}{\role{User}}
\newcommand{\guestrole}{\role{Guest}}

% ===== COMMANDES POUR LES APIs =====

% Méthodes HTTP
\newcommand{\httpget}{\httpmethod{GET}}
\newcommand{\httppost}{\httpmethod{POST}}
\newcommand{\httpput}{\httpmethod{PUT}}
\newcommand{\httpdelete}{\httpmethod{DELETE}}

% Codes de statut HTTP
\newcommand{\statusok}{\statuscode{200 OK}}
\newcommand{\statuscreated}{\statuscode{201 Created}}
\newcommand{\statusnotfound}{\statuscode{404 Not Found}}
\newcommand{\statuserror}{\statuscode{500 Internal Server Error}}

% Endpoints API
\newcommand{\apiendpoint}[1]{\endpoint{/api/v1/#1}}

% ===== COMMANDES POUR LA DOCUMENTATION =====

% Types de documentation
\newcommand{\apidoc}{\deliverable{API Documentation}}
\newcommand{\userdoc}{\deliverable{User Documentation}}
\newcommand{\techdoc}{\deliverable{Technical Documentation}}

% Outils de documentation
\newcommand{\swagger}{\software{Swagger}}
\newcommand{\postman}{\software{Postman}}
\newcommand{\jsdoc}{\software{JSDoc}}

% ===== COMMANDES POUR LES OUTILS =====

% Outils de développement
\newcommand{\vscode}{\software{Visual Studio Code}}
\newcommand{\git}{\software{Git}}
\newcommand{\github}{\software{GitHub}}
\newcommand{\docker}{\software{Docker}}
\newcommand{\kubernetes}{\software{Kubernetes}}

% Services cloud
\newcommand{\aws}{\software{Amazon Web Services}}
\newcommand{\azure}{\software{Microsoft Azure}}
\newcommand{\gcp}{\software{Google Cloud Platform}}

% ===== COMMANDES POUR LES DATES ET DURÉES =====

% Phases du projet
\newcommand{\phase}[1]{\milestone{Phase #1}}
\newcommand{\iteration}[1]{\milestone{Iteration #1}}
\newcommand{\release}[1]{\milestone{Release #1}}

% Durées typiques
\newcommand{\oneweek}{\duration{1 week}}
\newcommand{\twoweeks}{\duration{2 weeks}}
\newcommand{\onemonth}{\duration{1 month}}

% ===== COMMANDES POUR LES RÉFÉRENCES =====

% Références aux livrables
\newcommand{\deliverableref}[1]{Deliverable~\ref{deliv:#1}}
\newcommand{\milestoneref}[1]{Milestone~\ref{mile:#1}}
\newcommand{\requirementref}[1]{Requirement~\ref{req:#1}}

% Références aux tests
\newcommand{\testcaseref}[1]{Test Case~\ref{test:#1}}
\newcommand{\testsuiteref}[1]{Test Suite~\ref{suite:#1}}

% ===== COMMANDES CONDITIONNELLES =====

% Compilation conditionnelle
\newif\ifshowdetails
\showdetailstrue  % Afficher les détails par défaut

\newcommand{\detailsonly}[1]{\ifshowdetails#1\fi}
\newcommand{\summaryonly}[1]{\ifshowdetails\else#1\fi}

% Version du document
\newcommand{\version}[1]{\textit{\color{neutralmedium}v#1}}
\newcommand{\updated}[1]{\textit{\color{neutralmedium}Updated: #1}}

