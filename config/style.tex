% =====================================================
% CONFIGURATION AVANCÉE DU STYLE
% =====================================================

% ===== TYPOGRAPHIE PERSONNALISÉE =====

% Interligne (choix: \singlespacing, \onehalfspacing, \doublespacing)
\onehalfspacing  % Interligne 1.5 (recommandé pour les rapports)

% Espacement des paragraphes
\setlength{\parskip}{8pt plus 2pt minus 1pt}    % 8pt entre paragraphes
\setlength{\parindent}{0pt}                     % Pas d'indentation première ligne

% Marges personnalisées (optionnel - déjà défini dans la classe)
% \geometry{
%     top=2.5cm,
%     bottom=2.5cm,
%     left=3cm,
%     right=2.5cm,
%     headheight=15pt
% }

% ===== PERSONNALISATION DES TITRES =====

% Espacement autour des titres
\titlespacing*{\chapter}{0pt}{-10pt}{25pt}      % Chapitre: espace avant/après
\titlespacing*{\section}{0pt}{18pt}{12pt}       % Section: espace avant/après
\titlespacing*{\subsection}{0pt}{14pt}{8pt}     % Sous-section: espace avant/après
\titlespacing*{\subsubsection}{0pt}{12pt}{6pt}  % Sous-sous-section

% Couleurs des titres (personnalisable)
\titleformat{\chapter}[display]
    {\normalfont\huge\bfseries\color{brandprimary}}
    {\chaptertitlename\ \thechapter}{20pt}{\Huge\MakeUppercase}

\titleformat{\section}
    {\normalfont\Large\bfseries\color{brandsecondary}}
    {\thesection}{1em}{}

\titleformat{\subsection}
    {\normalfont\large\bfseries\color{neutraldark}}
    {\thesubsection}{1em}{}

% ===== CONFIGURATION DES LISTES =====

% Listes à puces - niveau 1
\setlist[itemize,1]{
    label=\textcolor{neutraldark}{\textbullet},
    leftmargin=20pt,
    itemsep=3pt,
    parsep=1pt,
    topsep=6pt,
    partopsep=0pt
}

% Listes à puces - niveau 2
\setlist[itemize,2]{
    label=\textcolor{neutraldark}{\textendash},
    leftmargin=35pt,
    itemsep=2pt,
    parsep=0pt,
    topsep=3pt
}

% Listes numérotées
\setlist[enumerate,1]{
    leftmargin=20pt,
    itemsep=3pt,
    parsep=1pt,
    topsep=6pt
}

% ===== STYLE DES TABLEAUX =====

% Espacement global des tableaux
\renewcommand{\arraystretch}{1.4}  % Hauteur des lignes
\setlength{\tabcolsep}{10pt}       % Espacement des colonnes

% Style des en-têtes de tableaux
\renewcommand{\tableheadercell}[1]{%
    \cellcolor{tableheadercolor}\textbf{\color{neutraldark}#1}%
}

% ===== CONFIGURATION DES FIGURES =====

% Positionnement par défaut des figures
\renewcommand{\topfraction}{0.9}       % Max 90% de la page pour figures en haut
\renewcommand{\bottomfraction}{0.9}    % Max 90% de la page pour figures en bas
\renewcommand{\textfraction}{0.1}      % Min 10% de la page pour le texte
\renewcommand{\floatpagefraction}{0.8} % Min 80% pour page de figures

% Style des légendes
\captionsetup[figure]{
    font={small,times},                    % Police sans-serif petite
    labelfont={bf,color=neutraldark},  % Label en gras coloré
    textfont={color=neutraldark},       % Texte en couleur neutre
    margin=20pt,                        % Marge des légendes
    skip=12pt,                          % Espace avec la figure
    position=bottom                     % Position sous la figure
}

\captionsetup[table]{
    font={small,times},
    labelfont={bf,color=neutraldark},
    textfont={color=neutraldark},
    margin=20pt,
    skip=8pt,
    position=top                        % Position au-dessus du tableau
}

% ===== STYLE DU CODE SOURCE =====

% Configuration globale pour tous les listings
\lstset{
    backgroundcolor=\color{codebackground},
    basicstyle=\footnotesize\ttfamily\color{neutraldark},
    commentstyle=\color{codecomment}\itshape,
    keywordstyle=\color{codekeyword}\bfseries,
    stringstyle=\color{codestring},
    numberstyle=\tiny\color{neutralmedium},
    numbers=left,
    numbersep=12pt,
    frame=single,
    rulecolor=\color{codeborder},
    frameround=tttt,
    framesep=10pt,
    xleftmargin=20pt,
    xrightmargin=10pt,
    breaklines=true,
    breakatwhitespace=false,
    tabsize=2,
    showspaces=false,
    showstringspaces=false,
    showtabs=false,
    captionpos=b
}

% ===== EN-TÊTES ET PIEDS DE PAGE =====

% Style pour les pages normales
\fancypagestyle{mainmatter}{
    \fancyhf{}
    \fancyhead[C]{\footnotesize\textcolor{neutralmedium}{\MakeUppercase{\@reporttitle}}}
    \fancyfoot[R]{\footnotesize\textcolor{neutralmedium}{%
        \leftmark \quad \textbf{|} \quad \textbf{\thepage}%
    }}
    \renewcommand{\headrulewidth}{0pt}
    \renewcommand{\footrulewidth}{0.5pt}
    \renewcommand{\footrule}{\color{neutrallight}\hrule width\headwidth height\footrulewidth}
}

% Style pour les pages de chapitre
\fancypagestyle{plain}{
    \fancyhf{}
    \fancyfoot[C]{\textcolor{neutraldark}{\thepage}}
    \renewcommand{\headrulewidth}{0pt}
    \renewcommand{\footrulewidth}{0pt}
}

% ===== PERSONNALISATION TABLE DES MATIÈRES =====

% Style du titre de la table des matières
% \renewcommand{\cfttoctitlefont}{\Large\bfseries\color{brandprimary}}

% Style des entrées de chapitre
\renewcommand{\cftchapfont}{\bfseries\color{neutraldark}}
\renewcommand{\cftchappagefont}{\bfseries\color{neutraldark}}
\renewcommand{\cftchapleader}{\cftdotfill{\cftsecdotsep}}

% Style des entrées de section
\renewcommand{\cftsecfont}{\color{neutraldark}}
\renewcommand{\cftsecpagefont}{\color{neutralmedium}}
